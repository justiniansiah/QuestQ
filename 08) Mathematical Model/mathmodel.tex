
\documentclass{article}
\usepackage{pagecolor}
\definecolor{ultramarine}{RGB}{254, 250, 214} 
\begin{document}
\pagecolor{ultramarine}
\title{Mathematical Modelling}
\author{QuestQ}
\date{}
\maketitle
\section*{Initial model \& Proposed alternative model}
In both inital and alternative models we tracked the following population parameters:
\begin{enumerate}
    \item Arrival times of customers entering the system. The arrival time of the $i$th customer is denoted by $A_i$.
    \item Departure times of customers leaving queuing system after collecting order. The departure time of the $i$th customer is denoted by $D_i$.
    \item Service times for customers in the system. The service time for the $i$th customer is given by $S_i$.
    \item Total number of customers that have entered/left the system is denoted by $n$. (Note: We do not account for customers leaving the system, without making an order hence the total number of arrivals = total number of departures)
\end{enumerate}
From the above, we derived the following:
\begin{enumerate}
    \item Inter-arrival times between customers entering the queuing system. The $j$th inter-arrival time is given by: \[A_i-A_{i-1}\]
    \item Inter-departure time between customers leaving the system. The $j$th inter-departure time is given by: \[D_i-D_{i-1}\]
    \item Lead time is traditionally interpreted as the total waiting time prior to making an order for the $i$th customer:
    \[D_i-A_i-S_i\]
    \item In our project, we denote lead time as the total waiting time since entering the system, till customers depart from their system after collecting an order. 
    The lead time of the $i$th customer is thus given by: \[D_i-A_i\]
    \item The total waiting time for all customers: \[\sum_i^n D_i-A_i\]
    \item Average waiting, during the 45 minute peak period:
    \[\frac{\sum_i^n D_i-A_i}{n}\]
    \item Rate of change in queue length (Expected number of people in the queue):
    \[\frac{dD}{dt}-\frac{dA}{dt}\]
\end{enumerate}
\end{document}